\chapter{Einleitung}
L�nge: ca. 1 - 5. Seiten\\\\

\noindent Aufbau:
\begin{itemize}
	\item Motivation der Arbeit /Problem
	\item Ansatz der L�sung / Ziele
	\item Struktur der Arbeit / Vorgehen
\end{itemize}
Die Einleitung dient dazu, beim Leser Interesse f�r das Thema der Arbeit zu wecken, das behandelte Problem aufzuzeigen und den zu seiner L�sung eingeschlagenen Weg zu beschreiben. In diesem Kapitel wird die mit dem Betreuer/Professor besprochene Aufgabenstellung herausgearbeitet und f�r einen potentiellen Leser "spannend" dargestellt.\\\\

\noindent Motivation:\\
In der Motivation wird dargestellt, wieso es notwendig ist, sich mit dem in der Arbeit identifizierten und behandelten Problem zu besch�ftigen. Zur Entwicklung der Motivation kann eine dem Leser bekannte Problematik aufgegriffen und dann die Problemstellung hieraus abgeleitet werden. Der Betreuer unterst�tzt die Entwicklung der Motivation, indem er bei der Einordnung der Arbeit in ein gr��eres Problemgebiet hilft.\\\\

\noindent H�ufige Fehler:
\begin{itemize}
	\item Zu allgemeine Motivation, Problemstellung und -abgrenzung. Die Problemstellung beginnt mit der Einordnung in ein thematisches Umfeld und enth�lt sowohl die in der Arbeit angegangenen Problempunkte, als auch weitere, nicht behandelte Problempunkte. Eine Negativabgrenzung verhindert, dass beim Leser sp�ter nicht erf�llte Erwartungen geweckt werden.
Der Betreuer unterst�tzt die Eingrenzung der Problemstellung, indem er Hinweise auf abzugrenzende Punkte bzw. auszuschlie�ende Punkte im Rahmen der Negativabgrenzung gibt.
\noindent H�ufige Fehler:
	\item Keine klare Problemstellung und -abgrenzung
	\item Fehlen der Negativabgrenzung
\end{itemize}

\noindent Ziel der Arbeit:\\
Mit dem Ziel der Arbeit wird der angestrebte L�sungsumfang festgelegt. An diesem Ziel wird die Arbeit gemessen.
Der Betreuer sorgt daf�r, dass das Ziel der Arbeit realisierbar und im Rahmen einer Diplomarbeit l�sbar ist.\\\\
\noindent H�ufige Fehler:
\begin{itemize}
	\item Kein klares Ziel
	\item Zu viele Ziele
\end{itemize}


\noindent Vorgehen:\\
Nachdem mit Problemstellung und Ziel gewisserma�en Anfangs- und Endpunkt der Arbeit beschrieben sind, wird hier der zur Erreichung des Ziels eingeschlagene Weg vorgestellt. Dazu werden typischerweise die folgenden Kapitel und ihr Beitrag zur Erreichung des Ziels der Arbeit kurz beschrieben. Die folgenden Kapitel sind ein {\em m�glicher} Aufbau, Abweichungen k�nnen durchaus notwendig sein. Zur Darstellung des Vorgehens kann eine grafische Darstellung sinnvoll sein, bei der die einzelnen L�sungsschritte und ihr Zusammenhang dargestellt werden.

