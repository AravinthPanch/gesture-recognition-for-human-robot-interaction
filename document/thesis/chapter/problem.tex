\chapter{Problemanalyse}
L�nge: ca. 5 - 15 Seiten\\\\
Das Kapitel Problemanalyse dient dazu, das in der Einleitung identifizierte und eingegrenzte Problem auf seine Ursachen zur�ckzuf�hren und so L�sungsm�glichkeiten zu entwickeln. Hierdurch wird die Problembezogenheit der entwickelten L�sung sichergestellt. Wenn m�glich, ist durch eine Literaturrecherche nachzuweisen, dass bisher keine geeigneten L�sungen existieren. Der Betreuer hilft bei der Entscheidung, ob die Problemanalyse ausreichend tief erfolgt ist. H�ufig f�hrt eine hinreichend genaue Problemanalyse zu pr�ziseren und damit k�rzeren L�sungskonzepten.\\\\

\noindent Kapitelzusammenfassung am Ende:\\
Der �bergang von der Problemanalyse zur Konzeptentwicklung stellt eine wichtige Nahtstelle innerhalb der Arbeit dar, da von einer betrachtend-analysierenden Perspektive auf eine konstruktiv-kreative Perspektive gewechselt wird. Daher empfiehlt es sich, an dieser Stelle die Ergebnisse der Problemanalyse zusammenzufassen.

