\section{NAO - The Humanoid Robot} 
NAO is an autonomous, programmable humanoid robot developed by Aldebaran Robotics. NAO Academics Edition was developed for universities and laboratories for research and education purposes. Follow subsections discuss briefly about the specifications of NAO as declared by Aldebaran Robotics.

\subsection{Construction}
NAO has a body with 25 degrees of freedom (DOF) whose key elements are electric motors and actuators. It has 48.6-watt-hour battery that provides NAO with 1.5 or more hours of autonomy, depending on usage. Table \ref{tb:nao} shows other specifications of NAO.

\subsection{Motion}
NAO's walking uses a simple dynamic model (linear inverse pendulum) and quadratic programming. It is stabilized using feedback from joint sensors. This makes walking robust and resistant to small disturbances, and torso oscillations in the frontal and lateral planes are absorbed. NAO can walk on a variety of floor surfaces, such as carpeted, tiled, and wooden floors. 

NAO's motion module is based on generalized inverse kinematics, which handles Cartesian coordinates, joint control, balance, redundancy, and task priority. This means that when asking NAO to extend its arm, it bends over because its arms and leg joints are taken into account. NAO will stop its movement to maintain balance.

In this thesis, we used locomotion control, stiffness control of Motion API to move NAO to a position in two dimensional space. Robot Posture API was also used to make the robot go to the predefined posture such as Stand, Sit and Crouch. Below given C++ programming function shows how NAO can be moved to another position at the given normalised velocity using Motion API.

\subsection{Audio}
NAO uses four directional microphones to detect sounds and equipped with a stereo broadcast system made up of 2 loudspeakers in its ears. NAOs voice recognition and text-to-speech capabilities allow it to communicate in 19 languages. 

In this thesis, we used Text-To-Speech API of NAO to say some words loud to communicate with the user.  Below given C++ programming function shows how NAO can say words given as strings.

\subsection{Vision}
Two identical video RGB cameras are located in the forehead of NAO. They provide up to 1280x960 resolution at 30 frames per second. NAO contains a set of algorithms for detecting and recognizing faces and shapes.

Skeletal points based gesture recognition needs three dimensional data of the human bone joints. However, sensors integrated with NAO couldnt provide precise three dimensional data for processing heavy algorithms to track human skeletal joints. 3D cameras such as Microsoft Kinect and Asus Xtion are used not only for gaming but also for analyzing 3D data, including algorithms for feature selection, scene analysis, motion tracking, skeletal tracking and gesture recognition \cite{12}. 

Asus Xtion PRO LIVE uses infrared sensors, adaptive depth detection technology, color image sensing and audio stream to capture a user's real-time image, movement, and voice, making user tracking more precise. 

Therefore in this thesis, we attempted to use Asus Xtion PRO LIVE as an external camera that was mounted on the head of NAO as shown in the figure \ref{fig:xtion}. 

\subsection{Computing}
NAO is equipped with Intel ATOM 1.6 GHz CPU in the head that runs a 32 bit Gentoo Linux to support Aldebarans proprietary middleware (NAOqi). The NAOqi Framework is the programming framework used to program Aldebaran robots. This framework allows homogeneous communication between different modules such as motion, audio, video. NAOqi executable which runs on the robot is a broker. The broker provides lookup services so that any module in the tree or across the network can find any method that has been advertised.

Computational limitations of NAOs CPU hinders us to build a real time gesture recognition based on human skeletal joints. Therefore, we used an off-board computer to execute the gesture recognition program and communicated with NAO via NAOqi proxies. 

\subsection{Networking}
Boost ASIO udp unicast socket wifi lan
