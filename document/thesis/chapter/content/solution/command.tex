\subsection{Command Module} Last but not the least module to complete the functionalities of our hand gesture recognition system is the Command module. All other modules which are described above need the Command module to interact with the robot.

Command module is developed in Python with WebSocket and NAOqi libraries. Python is a widely used general-purpose, high-level programming language. Its design philosophy emphasizes code readability, and its syntax allows programmers to express concepts in fewer lines of code than would be possible in languages such as C++ or Java. 

Command modules initiates the WebSocket Client and it connects to the Brain modules WebSocket Server at a given port number by loading the common configuration file. WebSocket client keeps the main thread run forever and it executes the respective call back handlers. When there is a new message, it calls the \textit{onMessage} handler and parses the received JSON data to a python object. Whenever gesture data is received, it is translated to a robotic speech and motion via NAOqi proxies.

We have used ALMotions Locomotion Control extensively to move the robot from one position to another based on the recognized hand gesture such as "Turn Left" or "Move Right". Additionally, Gesture-To-Gesture actions where a human hand gesture is translated to the robot hand gesture by using the Joint Control of ALMotion module.
