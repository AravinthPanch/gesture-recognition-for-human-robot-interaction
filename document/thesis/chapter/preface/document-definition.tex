\newcommand{\trtitle}{Gesture Recognition for Human-Robot Interaction: An approach based on skeletal points tracking using depth camera}

%%%%%%%%%%%%%%%%%%%%%%%%%%%%%%%%%%%%%%%%%%%%%%%%%%%%%%%%%%%%%%%%%%
\newcommand{\bmpstego}{stego}
%%%%%%%%%%%%%%%%%%%%%%%%%%%%%%%%%%%%%%%%%%%%%%%%%%%%%%%%%%%%%%%%%%


\documentclass[12pt,a4paper]{report}
%\documentclass[12pt,a4paper,onepage]{scrbook}

\usepackage[latin1]{inputenc}
\usepackage[T1]{fontenc}
\newcommand{\changefont}[3]{
\fontfamily{#1} \fontseries{#2} \fontshape{#3} \selectfont}

% Sprachen:
\usepackage[english]{babel} % Silbentrennung Deutsch neue Rechtschreibung
\selectlanguage{english}

\sloppy
%\usepackage{makeidx}
%\makeglossary
\makeindex


\usepackage{amsmath, marvosym} % Mathematik
\usepackage{times, url, geometry, amssymb, graphicx, booktabs}
\usepackage{fancyhdr} %Kopf- und Fu�zeilen
\usepackage[colorlinks,pagebackref,pdfpagelabels]{hyperref} %Hyperlinks zw. Textstellen
\usepackage[hyphenbreaks]{breakurl}
\usepackage{color} % Farben
\hypersetup{
	pdffitwindow=true,
	pdfmenubar=true,
	frenchlinks=false,
	colorlinks=false,
	bookmarksopen=true,
	bookmarksnumbered=true,
	pdfstartview=FitH,
	pdftitle = {\trtitle},
	pdfsubject = {\trtitle},
	pdfauthor = {Aravinth Panchadcharam},
	pdfkeywords = {Human-Robot Interaction (HRI), NAO, Computer Vision, Depth Camera, Hand Gesture, 3D hand based model, Skeleton tracking, Gesture Recognition, Sign Language Translation, Naive Bayes Classifier, Gesture Recognition Toolkit (GRT)},
	pdfcreator = {Adobe-Acrobat-Distiller},
	pdfproducer = {LaTeX using TeXStudio on Mac OSX}
}
\usepackage{subfigure} % mehrere Abbildungen nebeneinander/�bereinander
\usepackage{latexsym}

\geometry{a4paper,body={5.8in,9in}}
\setlength{\headheight}{15pt}

\usepackage{setspace} % 1,5 Zeilenabstand
\onehalfspacing
\setcounter{secnumdepth}{4}
\setcounter{tocdepth}{3} 


\clubpenalty = 10000 
\widowpenalty = 10000

% aller Bilder werden im Unterverzeichnis figures gesucht:
\graphicspath{{figures/}}

% Headers:
%\pagestyle{headings}
\pagestyle{fancy}
\pagestyle{headings}

% Literaturverzeichnis
\usepackage{bibgerm}
%\usepackage{natbib}
\bibliographystyle{gerunsrt} % Literaturangaben nach Auftreten sortieren %{gerplain}

%\usepackage{listings} % f�r Formatierung in Quelltexten
%\definecolor{grau}{gray}{0.25}
%\lstset{
%	extendedchars=true,
%	basicstyle=\scriptsize\ttfamily,
%	%basicstyle=\tiny\ttfamily,
%	tabsize=2,
%	keywordstyle=\textbf,
%	commentstyle=\color{grau},
%	stringstyle=\textit,
%	numbers=left,
%	numberstyle=\tiny,
%	% f�r sch�nen Zeilenumbruch
%	breakautoindent  = true,
%	breakindent      = 2em,
%	breaklines       = true,
%	postbreak        = ,
%	%prebreak         = \raisebox{-.8ex}[0ex][0ex]{\ensuremath{\lrcorner}},
%	prebreak         = \raisebox{-.8ex}[0ex][0ex]{\Righttorque},
%}

\usepackage{listings}
\usepackage{color}

\definecolor{dkgreen}{rgb}{0,0.6,0}
\definecolor{gray}{rgb}{0.5,0.5,0.5}
\definecolor{mauve}{rgb}{0.58,0,0.82}

\lstset{frame=tb,
	language=Java,
	aboveskip=3mm,
	belowskip=3mm,
	showstringspaces=false,
	columns=flexible,
	basicstyle={\small\ttfamily},
	numbers=none,
	numberstyle=\tiny\color{gray},
	keywordstyle=\color{blue},
	commentstyle=\color{dkgreen},
	stringstyle=\color{mauve},
	breaklines=true,
	breakatwhitespace=true,
	tabsize=3
}

\usepackage{caption}
\usepackage{multirow}
\usepackage{hyperref}
