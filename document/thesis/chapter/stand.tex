\chapter{Grundlagen und Stand der Forschung}
L�nge: ca. 5-10 Seiten\\\\

\noindent In diesem Kapitel werden f�r die weitere Arbeit wichtige Begriffe eingef�hrt. Dabei ist darauf zu achten, nur solche Inhalte in das Grundlagenkapitel aufzunehmen, die sp�ter auch verwendet werden (Problembezogenheit). Ebenso ist auf eine ausreichend Tiefe und vollst�ndige Darstellung der Grundlagen zu achten. Des Weiteren m�ssen verwandte schon vorhandene Arbeiten aus dem bearbeiteten Forschungsgebiet hier aufgef�hrt werden. Diese verwandten Arbeiten sind kurz zu analysieren. Wo immer m�glich sind Referenzen auf vorhandene Literatur einzusetzen. d.h. nur wenn von der Literatur abweichende Definitionen und Konzepte verwendet werden, ist eine ausf�hrliche Darstellung von Definitionen und Konzepten begr�ndet. Die Darstellung von Definitionen und Konzepten muss unbedingt homogen und widerspruchsfrei dargestellt werden. Keinesfalls d�rfen beispielsweise mehrere Definitionen des gleichen Begriffes nebeneinander gestellt werden, ohne dass eine begr�ndete Entscheidung f�r die letztlich in der Arbeit verwendete Definition getroffen wird. Ein weiterer Punkt ist die Vollst�ndigkeit der Grundlagen. So sollten alle m�glichen Merkmalskombinationen abgedeckt werden, was beispielsweise mit Hilfe einer Tabelle geschehen kann.

\noindent Der Betreuer gibt Hinweise auf relevante Literatur.\\\\

\noindent H�ufige Fehler sind:
\begin{itemize}
	\item Zuviel Grundlagen ohne Notwendigkeit f�r den Probleml�sungsprozess
	\item	Blo�es Nebeneinanderstellen von Definitionen ohne Auswahl einer f�r die Arbeit verbindlichen Definition
	\item Blo�es Nebeneinanderstellen von Definitionen ohne logischen Fluss
	\item	Unstrukturiertes Aneinanderreihen von Literaturzitaten ohne Beitrag zum Probleml�sungsprozess
\end{itemize}
