\chapter{Abstract}
Human\textendash robot interaction (HRI) has been a topic of both science fiction and academic speculation even before any robots existed. HRI research is  focusing to build an intuitive, and easy communication with the robot through speech, gestures, and facial expressions.
The use of hand gestures provides an attractive alternative to complex interface devices for human-computer interaction (HCI). In particular, visual interpretation of hand gestures can help in achieving the ease and naturalness desired for HCI. This has motivated a very active research area concerned with computer vision-based analysis and interpretation of hand gestures. 

In this thesis, we attempt to do the method of modeling, analyzing, and recognizing gestures. Important differences in the gesture interpretation approaches arise depending on whether moving model of the gesture or static model of the gesture is used.

We discuss directions of future research in gesture recognition, including its integration with other natural modes of humancomputer
interaction.

\chapter{Motivation}

Im Bereich ... (Warum muss es eine neue L�sung/ einen neuen Ansatz geben) 

\chapter{Zielsetzung}
Im Rahmen ... (Was will ich ueberhaupt mit meiner Abreit erreichen? Etwas verbessern, entwickeln, vergleichen...)


\chapter{Aufgabenpakete}
Ausgehend von der Zielbeschreibung werden folgende Arbeitspakete definiert:

\begin{itemize}
	\item Einarbeitung in das Themenbereich. Dazu geh�rt die Evaluierung relevanter Paper sowie das Ber�cksichtigen der bereits verf�gbaren Komponenten und Implemaentationen. Das Arbeitsergebnis ist eine Zusammenfassung relevanter Arbeiten.

	\item Entwicklung eines Ansatzes ...

	\item Analyse, Design und Entwurf einer Architektur

	\item Definition von geeigneten Testszenarien

	\item Umsetzung der Architektur, Implementierung

	\item Dokumentation der Architektur und des Programmcodes
\end{itemize}

	
\chapter{Zeitplan}
Die Bearbeitung dauert maximal 6 Monate. (Am besten eignet sich ein Gannt Diagramm)

\chapter{Organisatorisches}
\begin{itemize}
	\item Sprache der Diplomarbeit:
	\item Textverarbeitungssystem:
	\item Programmiersprache: Java 1.5
	\item Betreuer: 
	\item Gutachter: Prof. Dr. Sahin Albayrak, Dr.- Ing. Stefan Fricke
\end{itemize}

\chapter{Anhang}



