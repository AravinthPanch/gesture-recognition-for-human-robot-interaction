\chapter{Abstract}
Human-robot interaction (HRI) has been a topic of both science fiction and academic speculation even before any robots existed. HRI research is  focusing to build an intuitive, and easy communication with the robot through speech, gestures, and facial expressions. The use of hand gestures provides an attractive alternative to complex interfaced devices for HRI. In particular, visual interpretation of hand gestures can help in achieving the ease and naturalness desired for HRI. This has motivated a very active research area concerned with computer vision-based analysis and interpretation of hand gestures. Important differences in the gesture interpretation approaches arise depending on whether static model of the gesture or non-static model of the gesture is used. 

In this thesis, we attempt to do the method of modeling, analyzing, and recognizing gestures by using Computer Vision and Machine Learning techniques. Furthermore, Static (Gesture is formed by non-moving appearance of body parts) and non-static gestures (Gesture is formed by moving appearance of body parts), will be used to interact with robot and command the robot to execute certain actions.

We further hope to provide a platform to integrate Sign Language Translation to assist people with hearing and speech disabilities. However, further implementations and training data are needed to use this platform as a full fledged Sign Language Translator.

\chapter{Motivation}

Im Bereich ... (Warum muss es eine neue L�sung/ einen neuen Ansatz geben) 

Huge influence of computers in society has made smart devices, an important part of our lives. Availability and affordability of such devices motivated us to use such devices in our  day-to-day living. 

Interaction with such smart devices has been still been mostly done by displays, keyboards, mouse and touch interfaces. These devices have grown to be familiar but inherently limit the speed and naturalness with which we can interact with the computer. 

Smart devices include personal automatic and semi-automatic robots which are also playing a major role in a household. For an instance, Smart Vacuum Cleaner is an automatic robot that automatically cleans the floor and goes to its charging station without human interaction. Usage of robots for domestic and industrial purposes have been continuously increasing. Thus in recent years there has been a tremendous push in research toward an intuitive, and easy communication with the robot through speech, gestures, and facial expressions.

Tremendous progress has been made in speech recognition, and several commercially successful speech interfaces have been deployed. However, there has been an increased interest in recent years in trying to introduce other human-to-human communication modalities into HRI. This includes a class of techniques based on the movement of the human arm and hand, or hand gestures. The use of hand gestures provides an alternative mode of communication for Human-robot interaction (HRI) than the conventional interfaced devices.

Human hand gestures are a means of non-verbal interaction among people. They range from simple actions of using our hand to point at to the more complex ones that express our feelings and allow us to communicate with others. To exploit the use of gestures in HRI it is necessary to provide the means by which they can be interpreted by robots. The HCI interpretation of gestures requires that dynamic and/or static configurations of the human hand, arm, and even other parts of the human body, be measurable by the machine. 

\chapter{Zielsetzung}
Im Rahmen ... (Was will ich ueberhaupt mit meiner Abreit erreichen? Etwas verbessern, entwickeln, vergleichen...)


\chapter{Aufgabenpakete}
Ausgehend von der Zielbeschreibung werden folgende Arbeitspakete definiert:

\begin{itemize}
	\item Einarbeitung in das Themenbereich. Dazu geh�rt die Evaluierung relevanter Paper sowie das Ber�cksichtigen der bereits verf�gbaren Komponenten und Implemaentationen. Das Arbeitsergebnis ist eine Zusammenfassung relevanter Arbeiten.

	\item Entwicklung eines Ansatzes ...

	\item Analyse, Design und Entwurf einer Architektur

	\item Definition von geeigneten Testszenarien

	\item Umsetzung der Architektur, Implementierung

	\item Dokumentation der Architektur und des Programmcodes
\end{itemize}

	
\chapter{Zeitplan}
Die Bearbeitung dauert maximal 6 Monate. (Am besten eignet sich ein Gannt Diagramm)

\chapter{Organisatorisches}
\begin{itemize}
	\item Sprache der Diplomarbeit:
	\item Textverarbeitungssystem:
	\item Programmiersprache: Java 1.5
	\item Betreuer: 
	\item Gutachter: Prof. Dr. Sahin Albayrak, Dr.- Ing. Stefan Fricke
\end{itemize}

\chapter{Anhang}



