\chapter{L�sungskonzept}
L�nge: ca. 5 - 15 Seiten\\\\
Im L�sungskonzept wird auf konzeptueller Ebene der Weg zur L�sung der identifizierten Probleme beschrieben. Ausgangspunkt sind die Erkenntnisse der vorangegangenen Problemanalyse. Wichtig ist hierbei die Herausstellung des erzielten Neuigkeits- und Innovationswertes im Bezug auf den bisherigen Stand der Technik/Wissenschaft. Grundlage hierf�r ist ebenfalls die im vorangegangenen Kapitel durchgef�hrte Problemanalyse. Im L�sungskapitel werden noch keine umsetzungsspezifischen Details angef�hrt, dies ist Aufgabe des folgenden Kapitels. Eine typische Gliederung f�r die Darstellung des L�sungskonzepts ist das Aufgreifen der im vorangegangenen Kapitel identifizierten Problembereiche. Der Betreuer ber�t bei der Darstellung des L�sungskonzepts.\\\\

\noindent H�ufige Fehler:
\begin{itemize}
	\item L�sungskonzept passt nicht zum Ziel
	\item L�sungskonzept enth�lt Bestandteile der Umsetzung
\end{itemize}

\noindent Kapitelzusammenfassung am Ende:\\
Eine Zusammenfassung erleichtert es dem Leser, die erarbeitete L�sung zu erfassen.
